 \documentclass{article} %A4
\usepackage[a4paper,left=1.9cm, right=2.1cm,top = 1.2cm,bottom=2.3cm]{geometry}
\usepackage[utf8]{inputenc}%Umlaute
\usepackage[ngerman]{babel} %Texttrennung
\usepackage{graphicx}	%Grafiken
\usepackage{amssymb}
\usepackage{amsmath}
\usepackage{amsthm}
\usepackage{url}
\usepackage{listings}
 \usepackage{color}
\usepackage{hyperref}
\usepackage{framed}
\usepackage{algpseudocode}
\usepackage{tikz}

\usepackage[labelformat=empty]{caption}
\title{Zusammenfassung - Multi-Agenten-Systeme}
\author{
	Andreas Ruscheinski,
	Marc Meier
}


\begin{document}
\maketitle
\begin{framed}Korrektheit und Vollständigkeit der Informationen sind nicht gewährleistet.
Macht euch eigene Notizen oder ergänzt/korrigiert meine Ausführungen!
\end{framed}
\setcounter{tocdepth}{1}
\tableofcontents

\section{Einführung}
	\subsection{Definition}
	\begin{itemize}
		\item Ein Agent ist ein Computer System welches \textbf{selbstständig} Aktionen im Interesse des Benutzers ausführen kann.
		\item Ein Agent \textbf{befindet} sich in einer \textbf{dynamischen Umgebung} befindet mit welcher er interagiert.
		\item Ein Multi-Agenten-System besteht aus \textbf{meheren Agenten}, welcher \textbf{miteinander agieren}.
		\item In einem Multi-Agenten-System ist es notwending für \textbf{erfolgreiche Interaktion} dass Agenten miteinander \textbf{kooperieren},\textbf{ sich abstimmen} und miteinander \textbf{verhandeln} können.
	\end{itemize}
	\subsection{Eigenschaften}
	\begin{itemize}
		\item Jeder Agent hat \textbf{keine vollständigen Informationen} über die Umgebung
		\item Es gibt \textbf{keine globale Kontrolle} der Agenten
		\item Die Daten sind \textbf{dezentralisiert}
		\item Die Berechnung erfolgt \textbf{asynchron}
	\end{itemize}
	\subsection{Gründe für den Einsatz von MAS}
	\begin{itemize}
		\item Problem kann nicht zentralisiert gelöst werden da die \textbf{Ressourcen limitiert} sind
		\item \textbf{Reduktion der Ausfall-Wahrscheinlichkeit} in gegenüber einem zentralisierten System
		\item \textbf{Gewährleistung der inter-konnektion und inter-operation} von verschiedenen Systemen
		\item Lösung von Problemen welche eine\textbf{ Menge aus autonomen Komponenten behandeln}
	\end{itemize}
	\subsection{Konkrete Anwendungsgebiete}
	\begin{itemize}
		\item Clound-Management
		\item Ubiquitous Computing
		\item Grid-Software
		\item Spiele
		\item Verschiedene Gebiete der Industrie (Car-Assembly, Factory Management)
		\item Simulation
	\end{itemize}
\section{Rolle der Logik in MAS}
\end{document}