 \documentclass{article} %A4
\usepackage[a4paper,left=1.9cm, right=2.1cm,top = 1.2cm,bottom=2.3cm]{geometry}
\usepackage[utf8]{inputenc}%Umlaute
\usepackage[ngerman]{babel} %Texttrennung
\usepackage{graphicx}	%Grafiken
\usepackage{amssymb}
\usepackage{amsmath}
\usepackage{amsthm}
\usepackage{url}
\usepackage{listings}
 \usepackage{color}
\usepackage{hyperref}
\usepackage{framed}
\usepackage{algpseudocode}
\usepackage{tikz}

\usepackage[labelformat=empty]{caption}
\title{Zusammenfassung - Multi-Agenten-Systeme}
\author{
	Andreas Ruscheinski,
	Marc Meier
}


\begin{document}
\maketitle
\begin{framed}Korrektheit und Vollständigkeit der Informationen sind nicht gewährleistet.
Macht euch eigene Notizen oder ergänzt/korrigiert meine Ausführungen!
\end{framed}
\setcounter{tocdepth}{1}
\tableofcontents

\section{Einführung}
	\subsection{Definition}
	\begin{itemize}
		\item Ein Agent ist ein Computer System welches \textbf{selbstständig} Aktionen im Interesse des Benutzers ausführen kann.
		\item Ein Agent \textbf{befindet} sich in einer \textbf{dynamischen Umgebung} befindet mit welcher er interagiert.
		\item Ein Multi-Agenten-System besteht aus \textbf{meheren Agenten}, welcher \textbf{miteinander agieren}.
		\item In einem Multi-Agenten-System ist es notwending für \textbf{erfolgreiche Interaktion} dass Agenten miteinander \textbf{kooperieren},\textbf{ sich abstimmen} und miteinander \textbf{verhandeln} können.
	\end{itemize}
	\subsection{Eigenschaften}
	\begin{itemize}
		\item Jeder Agent hat \textbf{keine vollständigen Informationen} über die Umgebung
		\item Es gibt \textbf{keine globale Kontrolle} der Agenten
		\item Die Daten sind \textbf{dezentralisiert}
		\item Die Berechnung erfolgt \textbf{asynchron}
	\end{itemize}
	\subsection{Gründe für den Einsatz von MAS}
	\begin{itemize}
		\item Problem kann nicht zentralisiert gelöst werden da die \textbf{Ressourcen limitiert} sind
		\item \textbf{Reduktion der Ausfall-Wahrscheinlichkeit} in gegenüber einem zentralisierten System
		\item \textbf{Gewährleistung der inter-konnektion und inter-operation} von verschiedenen Systemen
		\item Lösung von Problemen welche eine\textbf{ Menge aus autonomen Komponenten behandeln}
	\end{itemize}
	\subsection{Konkrete Anwendungsgebiete}
	\begin{itemize}
		\item Clound-Management
		\item Ubiquitous Computing
		\item Grid-Software
		\item Spiele
		\item Verschiedene Gebiete der Industrie (Car-Assembly, Factory Management)
		\item Simulation
	\end{itemize}
\section{Rolle der Logik in MAS}
	\subsection{Gründe für Logik}
	\begin{itemize}
		\item Wissensbasis + Aktionen mit Voraussetzung und Auswirkung $\rightarrow$ Plan für Lösung des Problems
		\item Logik ist ein Framework für das Verstehen von Systemen
		\item Verifikation, Ausführungsspezifikation, Planung
	\end{itemize}
	\subsection{Logik-basierende Architektur}
	\begin{itemize}
		\item Grundidee: Beschreibung einer Regelmenge für die Beschreibung der besten Aktion für einen gegeben Zustand
		\item Bestandteile:
		\begin{itemize}
			\item $p$: eine Theorie (eine Menge von Regeln)
			\item $\Delta$: Datenbank mit den aktuellen Zustand der Welt
			\item $A$: eine Menge von Aktionen welcher ein Agent ausführen kann
			\item $\Delta\vdash_{p}\phi$: d.h. $\phi$ kann aus der $\Delta$ und $p$ abgeleitet werden, mit $\phi=$Do(a) können wir aus den aktuellen Zustand der Welt auf die bestmögliche Aktion logisch schließen
		\end{itemize}
		 \item Grundlegender Algorithmus (unabhänging von verwendeter Logik)
		 \begin{enumerate}
		 	\item see(s,p), generiert Beobachtung aus der neuen Welt
		 	\item next($\Delta$,p), update der Datenbank
		 	\item action($\Delta$), ermittelt die auszuführende Aktion aus der Datenbank, entweder ist die Aktion direkt beschrieben oder kann aus den Regeln abgeleitet werden kann
		 \end{enumerate}
	\end{itemize}
	\subsection{Modal Logik}
	\begin{itemize}
		\item Erlaubt Ausdrücke wie: wahrscheinlich wahr, geglaubt wahr, wahr in der Zukunft usw.
		\item Syntax:
		\begin{itemize}
			\item Prädikatenlogik mit Erweiterung
			\item Prop: eine menge von atomaren Formeln
			\item $\diamond$ p: möglicherweise p, manchmal p 
			\item $\square$p: immer p, notwendigerweise p
		\end{itemize}
		\item Semantik:
		\begin{itemize}
			\item Kripke-Struktur: $<W,R,\mu>$
			\begin{description}
				\item[$W$] eine Menge von Welten
				\item[$R$] eine Menge von binär Relationen, beschreiben den Übergang zwischen den Welten
				\item[$\mu$] Abbildungsfunktion welche jeder Welt Eigenschaften zuordnet ($\mu : W \rightarrow 2^{Prop}$)
			\end{description}
			\item Definition von $\diamond$ und $\square$ Operator auf Basis von Erreichbarkeit der Welten in einer Kripke-Struktur
			\item $\square p$: p ist wahr in allen Welten, welche von der aktuellen Welt erreichbar sind
			\item $\diamond p$: p ist wahr, wenn mindestens eine Welt erreichbar in welcher p wahr ist
			\item für R muss zusätzlich gelten:
			\begin{description}
				\item[reflexiv] für jedes $x \in W$ gilt $R(x,x)$
				\item[transitiv] für jedes $x,y,z \in W$ gilt $R(x,y) \wedge R(y,z) \implies R(x,z)$
				\item[seriell] für jedes $x \in W$ existiert ein $y$ so dass gilt $R(x,y)$
				\item[euklidisch] wenn für jedes $x,y,z \in W$ mit $R(x,y)$ und $R(x,z)$ gilt auch $R(y,z)$
			\end{description}
			\item Axiome:
			\begin{description}
				\item[$\square p \Rightarrow p$] Wenn immer $p$ gilt folgt daraus das aktuell p gilt
				\item[$\square p \Rightarrow \diamond p$] Wenn $p$ immer wahr ist, ist p auch in mindestens einer Welt wahr
				\item[$\square p \Rightarrow \square \square p$] Wenn $p$ ist immer wahr, ist p auch immer wahr wenn wir einen Übergang machen
				\item[$\diamond p \Rightarrow \square \diamond p$] Wenn $p$ in mindestens einer Welt wahr ist, ist p für immer Wahr wenn wir diese Welt erreicht haben
			\end{description}
			\item Anwendung der modal Logik auf Agenten durch Einführung von Indizes, welche entsprechend für Agent gelten
			\item Axiome und Agenten:
			\begin{description}
				\item[$K_{i}p \Rightarrow p$] Wenn Agent glaubt das p wahr ist, p ist auch in Wirklichkeit wahr
				\item[$K_{i}p \Rightarrow \neg K_{i}\neg p$] Wenn der Agent p glaubt, glaub er nicht die Negation
				\item[$K_{i}p \Rightarrow K_{i}K_{i}^p$] Wenn der Agent p glaubt, weiß er selbst dass er p glaubt
				\item[$\neg K_{i}\neg p \Rightarrow K_{i}\neg K_{i}\neg p$] Der Agent weiß, was er nicht weiß.
			\end{description}
		????
		\end{itemize}
	\end{itemize}
\section{Planning}
	\subsection{Einführung}
	\begin{itemize}
		\item Ziel 1: Intelligentes Verhalten ohne explizite Repräsentation des Wissens
		\item Ziel 2: Intelligenten Verhalten ohne abstraktes schließen über die Repräsentation des Wissens
		\item Idee 1: Echte Intelligenz gibt es nur ein einer Welt und nicht losgelöst von dieser wie in Theorem Beweisern und Expertensysteme
		\item Idee 2: Intelligentes Verhalten entsteht erst als Ergebnis der Interaktion mit der Umgeb
	\end{itemize}
\end{document}